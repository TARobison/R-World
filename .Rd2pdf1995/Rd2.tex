\documentclass[letterpaper]{book}
\usepackage[times,inconsolata,hyper]{Rd}
\usepackage{makeidx}
\usepackage[utf8,latin1]{inputenc}
% \usepackage{graphicx} % @USE GRAPHICX@
\makeindex{}
\begin{document}
\chapter*{}
\begin{center}
{\textbf{\huge Package `RWorld'}}
\par\bigskip{\large \today}
\end{center}
\begin{description}
\raggedright{}
\item[Type]\AsIs{Package}
\item[Title]\AsIs{What the package does (short line)}
\item[Version]\AsIs{1.0}
\item[Date]\AsIs{2017-10-26}
\item[Author]\AsIs{Who wrote it}
\item[Maintainer]\AsIs{Who to complain to }\email{yourfault@somewhere.net}\AsIs{}
\item[Description]\AsIs{More about what it does (maybe more than one line)}
\item[License]\AsIs{What license is it under?}
\item[RoxygenNote]\AsIs{6.0.1}
\end{description}
\Rdcontents{\R{} topics documented:}
\inputencoding{utf8}
\HeaderA{make.plants}{Generate plants to exist, reproduce, and compete within the terrain}{make.plants}
%
\begin{Description}\relax
Simulates plants on a terrain which is built in
terrain.r. Plants survive and reproduce in terrain that has a height 0 <= 
Ensure that you have the same length of vectors for params repro, survive and names!!
\end{Description}
%
\begin{Usage}
\begin{verbatim}
make.plants(terrain, survive = c(0.7, 0.85), repro = c(0.95, 0.55),
  names = NULL, timesteps = 50)
\end{verbatim}
\end{Usage}
%
\begin{Arguments}
\begin{ldescription}
\item[\code{survive}] The rate of survival of the listed plants. (default: .8, .65)
Is a vector of length equal to the number of listed plants

\item[\code{repro}] The rate of reproduction of the listed plants. (default: .4,.8)
Is a vector of length equal to the number of listed plants

\item[\code{names}] The names of listed plants. (defaul=NULL)
If no names are provided, plants will be assigned a letter in the alphabet 
Is a vector of length equal to the number of listed plants

\item[\code{timesteps}] The number of 'turns' that you want the simulation to exectute. (defaul: 50)
\end{ldescription}
\end{Arguments}
%
\begin{Value}
a plant matrix; empty cells contain only '', waterlogged cells contain Na,  
and cells containing plants will have the indicated (or assigned) name.
\end{Value}
%
\begin{Examples}
\begin{ExampleCode}
plants <- make.plants(make.terrain(6,15), c(.7,.85), repro=c(.95,.55), names=NULL, 50)
\end{ExampleCode}
\end{Examples}
\inputencoding{utf8}
\HeaderA{make.terrain}{Terrain with elivations represented by numerics in each cell}{make.terrain}
%
\begin{Description}\relax
Makes a matrix which is filled according to the diamond step algorithm.
Altitude is represented by values in each cell. Cells with NA are waterlogged.

This is a wrapper around \code{diamond.square.step} and \code{add.water}
\end{Description}
%
\begin{Usage}
\begin{verbatim}
make.terrain(n = 6, water = TRUE, sd = 15)
\end{verbatim}
\end{Usage}
%
\begin{Arguments}
\begin{ldescription}
\item[\code{n}] Determines the size of the grid. Grid will be 2\textasciicircum{}n +1 with a default of n=6

\item[\code{water}] a logical to indicate whether terrain lower than 0 should be underwater (default:TRUE)

\item[\code{sd}] the random noise to be added to each step of diamond.square.step
\end{ldescription}
\end{Arguments}
%
\begin{Value}
a terrain matrix; numeric elements indicate height, NAs indicate cells filled with water
\end{Value}
%
\begin{Examples}
\begin{ExampleCode}
terry <- make.terrain(4, 15)
image(terry)

\end{ExampleCode}
\end{Examples}
\printindex{}
\end{document}
